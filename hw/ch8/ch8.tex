% Physics Homework Template
% Useful for completing homework questions from the textbook.
\documentclass[11pt]{homework}
\usepackage{siunitx}
\usepackage{amsmath}
\usepackage{tikz}
\usepackage{pgfplots}
\pgfplotsset{compat=1.18}

\newcommand{\hwname}{Corbin Hibler}
\newcommand{\hwemail}{c-hibler@onu.edu}
\newcommand{\hwtype}{Ch. 8 HW}
\newcommand{\hwnum}{}
\newcommand{\hwclass}{PHYS 2311}
\newcommand{\hwlecture}{0}
\newcommand{\hwsection}{Z}

\begin{document}
\maketitle

% Problems
\renewcommand{\questiontype}{Problem}
\setcounter{questionCounter}{0}

% Problem 1
\question
\[
U = mgh
\]\[
U_i = (58)(9.8)(0) = \qty{0}{J}
\]\[
U_f= (58)(9.8)(4.0) = \qty{2300}{J}
\]\[
\Delta U = U_f - U_i = 2300 - 0 = \boxed{\qty{2300}{J}}
\]

% Problem 3
\setcounter{questionCounter}{2}
\question
\[
U_S = \frac{1}{2}kx^2
\]\[
45.0 = \frac{1}{2}(78.0)x^2
\]\[
x = \sqrt{\frac{2(45.0)}{78.0}} = \boxed{\qty{1.07}{m}}
\]

% Problem 4
\question
\begin{alphaparts}
    \questionpart
        \[
            \Delta U = mg(h_f-h_i) = (66.5)(9.8)(2660-1150) = \qty{984000}{J} = \boxed{\qty{984}{kJ}}
        \]
    \questionpart
        \[
            W = -\Delta U = \qty{-984}{kJ}
        \]
    \questionpart
        Yes, the work can be greater than this. The answer above assumes the perfect conditions given that the hiker experienced no air resistance, friction, and ignoring the path the hiker took, assuming he went straight to the final position. However, we know that the hiker is exerting a non-conservative force, meaning that the path is a factor. Therefore we know that the work is almost definitely greater than this.
\end{alphaparts}

% Problem 6
\setcounter{questionCounter}{5}
\question



% Problem 7
\question



% Problem 8
\question



% Problem 9
\question



% Problem 12
\setcounter{questionCounter}{11}
\question



% Problem 13
\question



% Problem 17
\setcounter{questionCounter}{16}
\question



% Problem 19
\setcounter{questionCounter}{18}
\question



% Problem 28
\setcounter{questionCounter}{27}
\question



% Problem 29
\question



% Problem 30
\question



% Problem 73
\setcounter{questionCounter}{72}
\question



\end{document}
