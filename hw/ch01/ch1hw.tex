% Physics Homework Template
% Useful for completing homework questions from the textbook.
\documentclass[11pt,largemargins]{homework}
\usepackage{siunitx}

\newcommand{\hwname}{Corbin Hibler}
\newcommand{\hwemail}{c-hibler@onu.edu}
\newcommand{\hwtype}{Ch. 1 HW}
\newcommand{\hwnum}{}
\newcommand{\hwclass}{PHYS 2311}
\newcommand{\hwlecture}{}
\newcommand{\hwsection}{}

\begin{document}
\maketitle

\renewcommand{\questiontype}{MisConcQ}
\setcounter{questionCounter}{1}
\question

$$ (3.84s)(37m/s)+(5.3s)(14.1m/s) $$
$$ 142.08m+74.73m $$
$$ 216.81m $$
\textbf{(e) 220m}

\question
\textbf{(e) Need more information.}

\question
\textbf{(b) 2}

\question
$$ 1.362 + 25.2 = 26.562 $$
$$ 26.6 $$
\textbf{(b) 3}

\question
\textbf{(b) how close a measurement is to the true value.}

\question
\textbf{(a) repeatability of a measurement, using a given instrument.}

\question
\textbf{(c) multiply by 9.}

\setcounter{questionCounter}{9}
\question
\textbf{(d) All of the above.}

\renewcommand{\questiontype}{Problem}
\setcounter{questionCounter}{0}
\question
  \begin{alphaparts}
    \questionpart 3
    \questionpart 4
    \questionpart 3
    \questionpart 1
    \questionpart 2
    \questionpart 4
    \questionpart 2
  \end{alphaparts}

\question
  \begin{alphaparts}
    \questionpart $ 5.859 \times 10^0 $
    \questionpart $ 2.18 \times 10^1 $
    \questionpart $ 6.8 \times 10^{-3} $
    \questionpart $ 3.2865 \times 10^2 $
    \questionpart $ 2.19 \times 10^{-1} $
    \questionpart $ 4.44 \times 10^2 $
  \end{alphaparts}

\question
  \begin{alphaparts}
    \questionpart 869,000
    \questionpart 9,100
    \questionpart 0.25
    \questionpart 476
    \questionpart 0.0000362
  \end{alphaparts}

\question
$$ \frac{0.35m}{3.25m} \times 100\% \approx \boxed{11\%} $$ 

\question
  \begin{alphaparts}
    \questionpart $ \frac{0.2s}{4.5s} \times 100\% \approx \boxed{4\%} $ 
    \questionpart $ \frac{0.2s}{45s} \times 100\% \approx \boxed{0.4\%} $ 
    \questionpart $ \frac{0.2s}{270s} \times 100\% \approx \boxed{0.07\%} $ 
  \end{alphaparts}

\question
$$  ( 9.2 \times 10^3 s ) + ( 6.3 \times 10^4 s ) + ( 0.008 \times 10^6 s ) $$
$$  ( 0.92 \times 10^4 s ) + ( 6.3 \times 10^4 s ) + ( 0.8 \times 10^4 s ) $$
$$ \boxed{8.0 \times 10^4 s}  $$

\question
$$ (4.079 \times 10^2 m) \times (0.057 \times 10^{-1} m) $$
$$ (407.9 m) \times (0.0057  m) $$
$$ \boxed{2.3m^2} $$

\question
$ \frac{0.01m^2}{1.27m^2} \times 100\% \approx \boxed{1\%} $

\setcounter{questionCounter}{13}
\question
  \begin{alphaparts}
    \questionpart \SI{0.2866}{\meter}
    \questionpart \SI{0.000064}{\volt}
    \questionpart \SI{0.430}{\gram}
    \questionpart \SI{0.0000000000472}{\second}
    \questionpart \SI{0.0000000225}{\meter}
    \questionpart \SI{2500000000}{\volt}
  \end{alphaparts}

\question
  \begin{alphaparts}
    \questionpart 3 megavolts
    \questionpart 2 micrometers
    \questionpart 5 kilodays
    \questionpart 18 hectobucks 
    \questionpart 900 nanoseconds
  \end{alphaparts}

\setcounter{questionCounter}{16}
\question
  \begin{alphaparts}
    \questionpart ratio of the surface area of Earth compared to the surface area of the Moon
      $$ A = 4 \pi r^2 $$
      $$ A_E = 4 \pi (6.38 \times 10^3 km)^2, \quad A_M = 4 \pi (1.74 \times 10^3 km)^2   $$ 
      $$ A_E = 4 \pi (40.7 \times 10^6 km^2), \quad A_M = 4 \pi (3.03 \times 10^6 km^2) $$
      $$ \frac{A_E}{A_M} = \frac{4 \pi (40.7 \times 10^6 km^2)}{4 \pi (3.03 \times 10^6 km^2)} $$
      $$ \frac{40.7 \times 10^6 km^2}{3.03 \times 10^6 km^2} $$
      $$ \boxed{13.4} $$
    \questionpart ratio of the volume of Earth compared to the volume of the Moon
      $$ V = \frac{4}{3} \pi r^3 $$
      $$ V_E = \frac{4}{3} \pi (6.38 \times 10^3 km)^3, \quad V_M = \frac{4}{3} \pi (1.74 \times 10^3 km)^3  $$
      $$ V_E = \frac{4}{3} \pi (260. \times 10^9 km^3), \quad V_M = \frac{4}{3} \pi (5.27 \times 10^9 km^3) $$
      $$ \frac{V_E}{V_M} = \frac{\frac{4}{3} \pi (260. \times 10^9 km^3)}{\frac{4}{3} \pi (5.27 \times 10^9 km^3)} $$
      $$ \frac{260. \times 10^9 km^3}{5.27 \times 10^9 km^3} $$
      $$ \boxed{49.3} $$
  \end{alphaparts}

\question
No, because:
$$ \frac{15m}{1s} \times \frac{3600s}{1h} \times \frac{1mi}{1609.34m} = \frac{33.554mi}{1h} $$
$$ 33.554 mi/h < 35 mi/h $$

\setcounter{questionCounter}{22}
\question
  \begin{alphaparts}
    \questionpart 
    $$ 1km \times \frac{100000cm}{1km} \times \frac{1in}{2.54cm} \times \frac{1mi}{63360} = 0.621 mi  $$
      $$ \boxed{\frac{0.621 mi/h}{1km/h}} $$
    \questionpart
      $$ 1m \times \frac{100cm}{1m} \times \frac{1in}{2.54cm} \times \frac{1ft}{12in} = 3.28 ft$$
      $$ \boxed{\frac{3.28 ft/s}{1m/s}} $$

    \questionpart
    $$ \frac{1km}{1h} \times \frac{1h}{3600s} \times \frac{1000m}{1km} = \frac{0.278m}{1s} $$
    $$ \boxed{\frac{0.278m/s}{1km/h}} $$
  \end{alphaparts}

\question
  \begin{alphaparts}
    \questionpart 
      $$ 1ft^2 \times \frac{1yd}{3ft} \times \frac{1yd}{3ft} = \boxed{\frac{1}{9}yd^2} $$
    \questionpart
    Using earlier conversion from meters to feet.
    $$ 1m^2 \times \frac{3.28ft}{1m} \times \frac{3.28ft}{1m} = \boxed{10.8 ft^2} $$
  \end{alphaparts}

\setcounter{questionCounter}{53}
\question
  $\frac{M}{L^3}$

\question
  \begin{alphaparts}
    \questionpart 
      $$ [v] = [At^3] - [Bt] $$
      $$ \frac{L}{T} = [A]T^3 - [B]T $$
      $$ \boxed{A = \frac{L}{T^4}, \quad B = \frac{L}{T^2}} $$
    \questionpart
    $$ \boxed{A = \frac{m}{s^4}, \quad B = \frac{m}{s^2}} $$
  \end{alphaparts}

\question
  \begin{alphaparts}
    \questionpart 
      No, this is incorrect.
      $$ x = [vt^2] + [2at] $$
      $$ x = \frac{LT^2}{T} + \frac{LT}{T^2}$$
      $$ x = LT + \frac{L}{T}$$
    \questionpart
      Yes, this is correct.
      $$ x = [v_0t] + [\frac{1}{2}at^2] $$
      $$ x = \frac{LT}{T} + \frac{LT^2}{T^2}$$
      $$ x = L + L$$
    \questionpart
      Yes, this is correct.
      $$ x = [v_0t] + [2at^2] $$
      $$ x = \frac{LT}{T} + \frac{LT^2}{T^2} $$
      $$ x = L + L$$
  \end{alphaparts}

\end{document}
