% Physics Homework Template
% Useful for completing homework questions from the textbook.
\documentclass[11pt]{homework}
\usepackage{siunitx}
\usepackage{amsmath}
\usepackage{tikz}
\usepackage{pgfplots}
\pgfplotsset{compat=1.18}

\newcommand{\hwname}{Corbin Hibler}
\newcommand{\hwemail}{c-hibler@onu.edu}
\newcommand{\hwtype}{Ch. 8 HW}
\newcommand{\hwnum}{}
\newcommand{\hwclass}{PHYS 2311}
\newcommand{\hwlecture}{0}
\newcommand{\hwsection}{Z}

\begin{document}
\maketitle

% Problems
\renewcommand{\questiontype}{Problem}
\setcounter{questionCounter}{0}

% Problem 1
\question
\[
U = mgh
\]\[
U_i = (58)(9.8)(0) = \qty{0}{J}
\]\[
U_f= (58)(9.8)(4.0) = \qty{2300}{J}
\]\[
\Delta U = U_f - U_i = 2300 - 0 = \boxed{\qty{2300}{J}}
\]

% Problem 3
\setcounter{questionCounter}{2}
\question
\[
U_S = \frac{1}{2}kx^2
\]\[
45.0 = \frac{1}{2}(78.0)x^2
\]\[
x = \sqrt{\frac{2(45.0)}{78.0}} = \boxed{\qty{1.07}{m}}
\]

% Problem 4
\question
\begin{alphaparts}
    \questionpart
        \[
            \Delta U = mg(h_f-h_i) = (66.5)(9.8)(2660-1150) = \qty{984000}{J} = \boxed{\qty{984}{kJ}}
        \]
    \questionpart
        \[
            W = -\Delta U = \qty{-984}{kJ}
        \]
    \questionpart
        Yes, the work can be greater than this. The answer above assumes the perfect conditions given that the hiker experienced no air resistance, friction, and ignoring the path the hiker took, assuming he went straight to the final position. However, we know that the hiker is exerting a non-conservative force, meaning that the path is a factor. Therefore we know that the work is almost definitely greater than this.
\end{alphaparts}

% Problem 6
\setcounter{questionCounter}{5}
\question
\begin{alphaparts}
    \questionpart
        \[
        U = mgh = (1.65)(9.8)(2.20) = \boxed{\qty{35.6}{J}}
        \]
    \questionpart
        \[
        U = mgh = (1.65)(9.8)(2.20-1.60)=\boxed{\qty{9.7}{}}
        \]
    \questionpart
    Following the principle that $W = -\Delta U$, we know that when lifting the book from the ground to \qty{2.20}{m} we can calculate the work done by the person from the change in potential energy of the book.
\end{alphaparts}


% Problem 7
\question
\[
\vec{F}(x)=-\frac{k}{x^3}\hat{i}
\]\[
U(x) = -\int \vec{F}(x) = -\int-\frac{k}{x^3}=k \int x^{-3}  = -\frac{1}{2}kx^{-2}+C=-\frac{k}{2x^2}+C
\]\[
U(\qty{2.0}{m})=0
\]\[
0=-\frac{k}{2(\qty{2.0}{m})^2}+C
\]\[
C=\frac{k}{2(\qty{2.0}{m})^2}=\frac{k}{2(\qty{4.0}{m^2})}=\frac{k}{\qty{8}{m^2}}
\]\[
U(x)=\boxed{-\frac{k}{2x^2}+\frac{k}{\qty{8}{m^2}}}
\]

% Problem 8
\question
\[
\vec{F}(x,y,z)=-\frac{\partial U}{\partial x}(\hat{i})-\frac{\partial U}{\partial y}(\hat{j})-\frac{\partial U}{\partial z}(\hat{k})
\]
\[
=-\frac{\partial 3x^2+2xy+4y^2z}{\partial x}(\hat{i})-\frac{\partial 3x^2+2xy+4y^2z}{\partial y}(\hat{j})-\frac{\partial 3x^2+2xy+4y^2z}{\partial z}(\hat{k})
\]
\[
=-\frac{\partial 3x^2+2xy}{\partial x}(\hat{i})-\frac{\partial 2xy+4y^2z}{\partial y}(\hat{j})-\frac{\partial 4y^2z}{\partial z}(\hat{k})
\]
\[
    =\boxed{-(6x+2y) \hat{i} - (2x + 8yz)\hat{j}-(4y^2)\hat{k}}
\]


% Problem 9
\question
\[
    \vec{F}=(-kx+ax^3+bx^4)\hat{i}
\]
\begin{alphaparts}
    \questionpart
        Yes, the force is conservative. If we use the definition that the gradient $\nabla F = 0$ for conservative forces, we know that since this function only contains an $x$ component, it is conservative.
    \questionpart
    \[
    U(x)=-\int  F(x)\,dx
    \]
    \[
        =-\int(-kx+ax^3+bx^4)\, dx = \int(kx-ax^3-bx^4)\, dx = \frac{1}{2}kx^2 - \frac{1}{4}ax^4-\frac{1}{5}bx^5 + C
    \]
    \[
        U(x) = \boxed{\frac{1}{2}kx^2 - \frac{1}{4}ax^4-\frac{1}{5}bx^5 + C}
    \]
\end{alphaparts}


% Problem 12 (example done in class)
\setcounter{questionCounter}{11}
\question
\[
    \vec{v_i} = \qty{5.0}{m/s}, \quad \vec{v_f}=\qty{0}{m/s}, \quad h_i = \qty{0}{m}
\]
\[
\Delta K = - \Delta U
\]
\[
    \frac{1}{2}m(v_f^2-v_i^2)=-mg(h_f-h_i)
\]
\[
    h_f=-\frac{v_i^2-v_i^2}{2g}+h_i=-\frac{0-(5.0)^2}{2(9.8)}+0=\boxed{\qty{1.28}{m}}
\]
The length of the vine does not matter as it is not included in our calculation. However, if the vine is too short, then Jane would eventually end up wrapping around going downward. So the length must be $L>\frac{\Delta h}{2}$.


% Problem 13
\question
\[
\Delta K = - \Delta U
\]\[
\frac{1}{2}m(v_f^2-v_i^2)=-mg(h_f-h_i)
\]
\[
    v_i = \sqrt{2g(h_f-h_i)-v_f^2} = \sqrt{2(9.8)(1.22)} = \boxed{\qty{4.89}{m/s}}
\]


% Problem 17
\setcounter{questionCounter}{16}
\question
\[
    \vec{v}=\qty{85}{km/h}, \quad m=\qty{1400}{kg}, \quad x=\qty{2.2}{m}
\]\[
U_S = \frac{1}{2}kx^2
\]
\[
\Delta K = - \Delta U
\]
\[
    \frac{1}{2}mv^2 = \frac{1}{2}kx^2
\]
\[
mv^2 = kx^2 
\]
\[
    k=\frac{mv^2}{x^2}=\frac{(1400)(23.6)^2}{(2.2)^2}=\boxed{\qty{1.6e5}{N/m}}
\]



% Problem 19
\setcounter{questionCounter}{18}
\question
\begin{alphaparts}
    % Part A
    \questionpart
\[
    U_S = -\frac{1}{2}kx^2 = -\frac{1}{2}(875)(0.220)^2 = \qty{-21.175}{J}
\]
\[
\Delta K = - \Delta U
\]
\[
\Delta K = 21.175
\]
\[
    \frac{1}{2}mv^2 = 21.175
\]
\[
    v = \sqrt{\frac{2(21.175)}{m}} =  \sqrt{\frac{2(21.175)}{0.380}} = \boxed{\qty{10.4}{m/s}}
\]
    % Part B
    \questionpart
    \[
    \Delta K = \Delta U
    \]\[
    \frac{1}{2}mv^2 = mgh
    \]
    \[
    \frac{mv^2}{2mg}=h
    \]
    \[
        h = \frac{v^2}{2g}=\frac{(10.4)^2}{2(9.8)}
    = \boxed{\qty{5.69}{m}}
    \]
    

\end{alphaparts}

% Problem 28 (example done in class)
\setcounter{questionCounter}{27}
\question
\[
K_i + U_i = K_f + U_f + E_{\text{therm}}
\]
\[
    0 +mgh = \frac{1}{2}mv^2 + 0 + E_{\text{therm}}
\]
\[
    mgh_f - \frac{1}{2}mv_f^2 = E_{\text{therm}}
\]
\[
    E_{\text{therm}}=(16.0)(9.8)(2.20)-\frac{1}{2}(16.0)(1.15)^2 = \boxed{\qty{334}{J}}
\]

% Problem 29
\question
\begin{alphaparts}
\questionpart
\[
    \mu_k = 0.090, \quad v_i = \qty{0}{m/s}, \quad d = \qty{85}{m}, \quad h_i = 85\sin(28) \unit{m}, \quad h_f = \qty{0}{m}
\]
\[
K_i + U_i = K_f + U_f + f_kd 
\]
\[
    0 + mgh_i = \frac{1}{2}mv_f^2 + 0 + \mu_kmg\cos(\theta) d 
\]
\[
    gh_i - \mu_kg\cos(\theta)d = \frac{1}{2}v_f^2
\]
\[
    v_f=\sqrt{2g(h_i-\mu_k\cos\theta d)}=\sqrt{2(9.8)(85\sin(28)-(0.090)(85)(\cos(28))} = \boxed{\qty{25.49}{m/s}}
\]
\questionpart
\[
    v_i = \qty{25}{m/s}, \quad v_f = \qty{0}{m/s}, \quad h_f, h_i = \qty{0}{m}, \quad \mu_k = 0.090
\]
\[
K_i + U_i = K_f + U_i + f_kd
\]
\[
    \frac{1}{2}mv_i^2+0=0+0+\mu_kmgd
\]
\[
    d = \frac{v_i^2}{2\mu_kg}=\frac{(25.49)^2}{2(0.090)(9.8)}=\boxed{\qty{368}{m}}
\]
\end{alphaparts}

% Problem 30
\question
\begin{alphaparts}
    % Part A
    \questionpart
    \[
    h_i = \qty{14.0}{m}, \quad h_f = \qty{0}{m}, \quad v_i = \qty{0}{m/s}, \quad m = \qty{0.145}{kg}
    \]
    \[
    K_i + U_i = K_f + U_f
    \]
    \[
        0 + mgh_i = \frac{1}{2}mv_f^2 + 0
    \]
    \[
        v_f = \sqrt{2gh_i}=\sqrt{2(9.8)(14.0)}=\boxed{\qty{16.57}{m/s}}
    \]
    
    
    
    % Part B
    \questionpart
    \[
        K_i + U_i = K_f + U_f + W_{air}
    \]
    \[
        0 + mgh_i = \frac{1}{2}mv_f^2 + 0 + W_{air}
    \]
    \[
        mgh_i - \frac{1}{2}mv_f^2 = W_{air}
    \]
    \[
        W_{air} = m(gh_i-\frac{1}{2}v_f^2)=(0.145)((9.8)(14.0)-\frac{1}{2}(12.5)^2) = \qty{8.57}{J}
    \]
    \[
        F = \frac{W}{d} = \frac{8.57}{14.0} = \boxed{\qty{0.61}{N}}
    \]
\end{alphaparts}

% Problem 73
\setcounter{questionCounter}{72}
\question
\begin{alphaparts}
    % Part A
    \questionpart
    Where $\frac{dU(x)}{dx} < 0 $\\
    $[0,3), (6,9)$

    % Part B
    \questionpart
    Where $\frac{dU(x)}{dx} = 0$\\
    Minimum is $x=\qty{3}{m},\qty{6}{m},\qty{9}{m}$

    % Part C
    \questionpart
    Where the slope is the highest.\\
    Maximum is $x\approx \qty{4}{m}$

\end{alphaparts}


\end{document}
